\documentclass[10pt]{article}

\usepackage{fancyhdr}
\usepackage[includeheadfoot,left=0.5in, right=0.5in, top=0.5in, bottom=0.5in]{geometry}
\usepackage{lastpage}
\usepackage{extramarks}
\usepackage[usenames,dvipsnames]{color}
\usepackage{graphicx}
\usepackage{listings}
\usepackage{courier}
\usepackage{float}
\usepackage{url}
\usepackage{subfigure}
\usepackage{varwidth}
\usepackage{caption}
\usepackage{multirow}
\usepackage[pdfborder={0 0 0}]{hyperref}
\usepackage[compact,small]{titlesec}
\usepackage{microtype}
\usepackage{verbatim}
\usepackage{booktabs}
\usepackage{indentfirst}
\usepackage{pgffor}

\parskip = 0.5\baselineskip
\setlength{\belowcaptionskip}{-\baselineskip}

\captionsetup{font=scriptsize}
\captionsetup{labelfont=bf}

\pagestyle{fancy}
\rhead{Samir Silbak}
\lhead{EECS6083: Compiler Theory}
\rfoot{Page\ \thepage\ of \protect\pageref{LastPage}}
\cfoot{}
\renewcommand\headrulewidth{0.4pt}
\renewcommand\footrulewidth{0.4pt}

\setlength\parindent{0pt} % Removes all indentation from paragraphs

\title{
    \vspace{2in}
    \textmd{\textbf{EECS6083: Compiler Theory}}\\
    \vspace{4in}
}
\author{\textbf{Samir Silbak}}

\begin{document}
\maketitle
\newpage
\section{Project Description}
For this project we were required to implement a recursive decent parser (LL(1)
compiler). We start off by creating a scanner which communicates with the
parser. The scanner feeds each token to the parser. The parser will then make
sure syntax and semantics are correct. Once this is ensured, we will then
generate C code. This generated C code is then compiled and linked with
runtime functions. The language for which the compiler is written in is Python.
The source code can be found in the \texttt{code} directory.

\section{Usage}
\texttt{compiler.py} was written to combine all of the source code together,
hence this is the top level module. Running this Python program will generate,
compile and run the executable. There are several tests that have been added in
the \texttt{tests} directory. An example of using the program would be like so:

\begin{verbatim}
$ ./compiler.py tests/factorial.src
3628800
\end{verbatim}

If the program is run without any arguments, the generated output will be
shown:

\begin{verbatim}
usage: compiler.py [-h] filename
compiler.py: error: too few arguments
\end{verbatim}

Passing the \texttt{h} flag will let the user know that the program takes only
\texttt{src} file extensions:

\begin{verbatim}
./compiler.py -h
usage: compiler.py [-h] filename

EECS 6083 Compiler

positional arguments:
  filename    input .src file

optional arguments:
  -h, --help  show this help message and exit
\end{verbatim}

\end{document}
